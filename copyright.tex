\mbox{}
\vskip88mm
\headrule
\vskip9mm
{\scriptsize
{\setlength{\baselineskip}
{0.6\baselineskip}
\sffamily
\textbf{APhEx.it \`e un periodico elettronico, registrazione n$^{\circ}$ ISSN
2036-9972. Il copyright degli articoli \`e libero. Chiunque pu\`o riprodurli.
Unica condizione: mettere in evidenza che il testo riprodotto \`e tratto da}
\underline{www.aphex.it} \\

Condizioni per riprodurre i materiali $\rightarrow$ Tutti i materiali, i dati e
le informazioni pubblicati all'interno di questo sito web sono ``no copyright'',
nel senso che possono essere riprodotti, modificati, distribuiti, trasmessi,
ripubblicati o in altro modo utilizzati, in tutto o in parte, senza il
preventivo consenso di APhEx.it, \ul{a condizione che tali utilizzazioni
avvengano per finalit\`a di uso personale, studio, ricerca o comunque non
commerciali e che sia citata la fonte attraverso la seguente dicitura, impressa
in caratteri ben visibili: ``www.aphex.it''}. Ove i materiali, dati o
informazioni siano utilizzati in forma digitale, la citazione della fonte
dovr\`a essere effettuata in modo da consentire \ul{un collegamento ipertestuale
(link) alla home page www.aphex.it o alla pagina dalla quale i materiali, dati o
informazioni sono tratti}. In ogni caso, dell'avvenuta riproduzione, in forma
analogica o digitale, dei materiali tratti da www.aphex.it dovr\`a essere data
tempestiva comunicazione al seguente indirizzo (redazione@aphex.it), allegando,
laddove possibile, copia elettronica dell'articolo in cui i materiali sono stati
riprodotti. \\

In caso di citazione su materiale cartaceo \`e possibile citare il materiale
pubblicato su APhEx.it come una rivista cartacea, indicando il numero in cui \`e
stato pubblicato l'articolo e l'anno di pubblicazione riportato anche
nell’intestazione del pdf. Esempio: Autore, \emph{Titolo,
{\lcap}www.aphex.it{\rcap},} 1 (2010).
\par}
}
\vskip7mm
\headrule
